\problemname{Alfametik}

\begin{figure}
   \centering
	\includegraphics{alfametik}
\end{figure}

En alfametik är ett matematiskt pussel där en antal ord är ordnade på den form man använder när man utför en addition "för hand", och där det går ut på att ersätta bokstäverna med siffror så att resultatet blir en giltig aritmetisk summa. Den första moderna alfametiken, publicerad av H E Dudeney i juli 1924, visas i figuren ovan och har lösningen $9567 + 1085 = 10652$.

Det finns två tämligen självklara regler för en alfametik: 

\begin{itemize}

\item Det ska finnas ett 1 - 1 förhållande mellan bokstäverna och siffrorna i alfametiken. Det vill säga, samma bokstav står alltid för samma siffra, och samma siffra är alltid representerad av samma bokstav.

\item Siffran 0 får inte förekomma längst till vänster i termerna eller summan.

\end{itemize}

\section*{Indata}
På första raden står antalet termer, ett tal mellan 2 och 10. Därefter kommer termerna, en på varje rad, och sist summan. Inget av dessa "tal" kommer att vara längre än 8 bokstäver. Bokstäverna kommer att vara uteslutande versaler, A till Z. 

\section*{Utdata}
Programmet ska ange lösningen till problemet genom att skriva ut termerna (i samma ordning som i indatan) samt summan på separata rader. Du kan anta att det existerar exakt en lösning. 


\section*{Poängsättning}
Din lösning kommer att testas på flera testfall. För att få 100 poäng så måste du klara alla testfall.

